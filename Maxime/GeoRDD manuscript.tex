
% Inherit from the specified cell style.

% Default to the notebook output style

    


% Inherit from the specified cell style.




    
\documentclass[letter]{article}

    
    
% \usepackage{stix}
\usepackage{bm}
\usepackage[T1]{fontenc}
% Nicer default font than Computer Modern for most use cases
\usepackage{palatino}

% Basic figure setup, for now with no caption control since it's done
% automatically by Pandoc (which extracts ![](path) syntax from Markdown).
\usepackage{graphicx}
% We will generate all images so they have a width \maxwidth. This means
% that they will get their normal width if they fit onto the page, but
% are scaled down if they would overflow the margins.
\makeatletter
\def\maxwidth{\ifdim\Gin@nat@width>\linewidth\linewidth
\else\Gin@nat@width\fi}
\makeatother
\let\Oldincludegraphics\includegraphics
% Set max figure width to be 80% of text width, for now hardcoded.
\renewcommand{\includegraphics}[1]{\Oldincludegraphics[width=.8\maxwidth]{#1}}
% Ensure that by default, figures have no caption (until we provide a
% proper Figure object with a Caption API and a way to capture that
% in the conversion process - todo).
\usepackage{caption}
\DeclareCaptionLabelFormat{nolabel}{}
\captionsetup{labelformat=nolabel}

\usepackage{adjustbox} % Used to constrain images to a maximum size 
\usepackage{xcolor} % Allow colors to be defined
\usepackage{enumerate} % Needed for markdown enumerations to work
\usepackage{geometry} % Used to adjust the document margins
\usepackage{amsmath} % Equations
\usepackage{amssymb} % Equations
\usepackage{textcomp} % defines textquotesingle
% Hack from http://tex.stackexchange.com/a/47451/13684:
\AtBeginDocument{%
    \def\PYZsq{\textquotesingle}% Upright quotes in Pygmentized code
}
\usepackage{upquote} % Upright quotes for verbatim code
\usepackage{eurosym} % defines \euro
\usepackage[mathletters]{ucs} % Extended unicode (utf-8) support
\usepackage[utf8x]{inputenc} % Allow utf-8 characters in the tex document
\usepackage{fancyvrb} % verbatim replacement that allows latex
\usepackage{grffile} % extends the file name processing of package graphics 
                     % to support a larger range 
% The hyperref package gives us a pdf with properly built
% internal navigation ('pdf bookmarks' for the table of contents,
% internal cross-reference links, web links for URLs, etc.)
\usepackage{hyperref}
\usepackage{longtable} % longtable support required by pandoc >1.10
\usepackage{booktabs}  % table support for pandoc > 1.12.2
\usepackage[normalem]{ulem} % ulem is needed to support strikethroughs (\sout)
                            % normalem makes italics be italics, not underlines



    
    
    
    % Colors for the hyperref package
    \definecolor{urlcolor}{rgb}{0,.145,.698}
    \definecolor{linkcolor}{rgb}{.71,0.21,0.01}
    \definecolor{citecolor}{rgb}{.12,.54,.11}

    % ANSI colors
    \definecolor{ansi-black}{HTML}{3E424D}
    \definecolor{ansi-black-intense}{HTML}{282C36}
    \definecolor{ansi-red}{HTML}{E75C58}
    \definecolor{ansi-red-intense}{HTML}{B22B31}
    \definecolor{ansi-green}{HTML}{00A250}
    \definecolor{ansi-green-intense}{HTML}{007427}
    \definecolor{ansi-yellow}{HTML}{DDB62B}
    \definecolor{ansi-yellow-intense}{HTML}{B27D12}
    \definecolor{ansi-blue}{HTML}{208FFB}
    \definecolor{ansi-blue-intense}{HTML}{0065CA}
    \definecolor{ansi-magenta}{HTML}{D160C4}
    \definecolor{ansi-magenta-intense}{HTML}{A03196}
    \definecolor{ansi-cyan}{HTML}{60C6C8}
    \definecolor{ansi-cyan-intense}{HTML}{258F8F}
    \definecolor{ansi-white}{HTML}{C5C1B4}
    \definecolor{ansi-white-intense}{HTML}{A1A6B2}

    % commands and environments needed by pandoc snippets
    % extracted from the output of `pandoc -s`
    \providecommand{\tightlist}{%
      \setlength{\itemsep}{0pt}\setlength{\parskip}{0pt}}
    \DefineVerbatimEnvironment{Highlighting}{Verbatim}{commandchars=\\\{\}}
    % Add ',fontsize=\small' for more characters per line
    \newenvironment{Shaded}{}{}
    \newcommand{\KeywordTok}[1]{\textcolor[rgb]{0.00,0.44,0.13}{\textbf{{#1}}}}
    \newcommand{\DataTypeTok}[1]{\textcolor[rgb]{0.56,0.13,0.00}{{#1}}}
    \newcommand{\DecValTok}[1]{\textcolor[rgb]{0.25,0.63,0.44}{{#1}}}
    \newcommand{\BaseNTok}[1]{\textcolor[rgb]{0.25,0.63,0.44}{{#1}}}
    \newcommand{\FloatTok}[1]{\textcolor[rgb]{0.25,0.63,0.44}{{#1}}}
    \newcommand{\CharTok}[1]{\textcolor[rgb]{0.25,0.44,0.63}{{#1}}}
    \newcommand{\StringTok}[1]{\textcolor[rgb]{0.25,0.44,0.63}{{#1}}}
    \newcommand{\CommentTok}[1]{\textcolor[rgb]{0.38,0.63,0.69}{\textit{{#1}}}}
    \newcommand{\OtherTok}[1]{\textcolor[rgb]{0.00,0.44,0.13}{{#1}}}
    \newcommand{\AlertTok}[1]{\textcolor[rgb]{1.00,0.00,0.00}{\textbf{{#1}}}}
    \newcommand{\FunctionTok}[1]{\textcolor[rgb]{0.02,0.16,0.49}{{#1}}}
    \newcommand{\RegionMarkerTok}[1]{{#1}}
    \newcommand{\ErrorTok}[1]{\textcolor[rgb]{1.00,0.00,0.00}{\textbf{{#1}}}}
    \newcommand{\NormalTok}[1]{{#1}}
    
    % Additional commands for more recent versions of Pandoc
    \newcommand{\ConstantTok}[1]{\textcolor[rgb]{0.53,0.00,0.00}{{#1}}}
    \newcommand{\SpecialCharTok}[1]{\textcolor[rgb]{0.25,0.44,0.63}{{#1}}}
    \newcommand{\VerbatimStringTok}[1]{\textcolor[rgb]{0.25,0.44,0.63}{{#1}}}
    \newcommand{\SpecialStringTok}[1]{\textcolor[rgb]{0.73,0.40,0.53}{{#1}}}
    \newcommand{\ImportTok}[1]{{#1}}
    \newcommand{\DocumentationTok}[1]{\textcolor[rgb]{0.73,0.13,0.13}{\textit{{#1}}}}
    \newcommand{\AnnotationTok}[1]{\textcolor[rgb]{0.38,0.63,0.69}{\textbf{\textit{{#1}}}}}
    \newcommand{\CommentVarTok}[1]{\textcolor[rgb]{0.38,0.63,0.69}{\textbf{\textit{{#1}}}}}
    \newcommand{\VariableTok}[1]{\textcolor[rgb]{0.10,0.09,0.49}{{#1}}}
    \newcommand{\ControlFlowTok}[1]{\textcolor[rgb]{0.00,0.44,0.13}{\textbf{{#1}}}}
    \newcommand{\OperatorTok}[1]{\textcolor[rgb]{0.40,0.40,0.40}{{#1}}}
    \newcommand{\BuiltInTok}[1]{{#1}}
    \newcommand{\ExtensionTok}[1]{{#1}}
    \newcommand{\PreprocessorTok}[1]{\textcolor[rgb]{0.74,0.48,0.00}{{#1}}}
    \newcommand{\AttributeTok}[1]{\textcolor[rgb]{0.49,0.56,0.16}{{#1}}}
    \newcommand{\InformationTok}[1]{\textcolor[rgb]{0.38,0.63,0.69}{\textbf{\textit{{#1}}}}}
    \newcommand{\WarningTok}[1]{\textcolor[rgb]{0.38,0.63,0.69}{\textbf{\textit{{#1}}}}}
    
    
    % Define a nice break command that doesn't care if a line doesn't already
    % exist.
    \def\br{\hspace*{\fill} \\* }
    % Math Jax compatability definitions
    \def\gt{>}
    \def\lt{<}
    % Document parameters
    \title{GeoRDD manuscript}
    
    \author{Maxime Rischard}
    

    % Pygments definitions
    
\makeatletter
\def\PY@reset{\let\PY@it=\relax \let\PY@bf=\relax%
    \let\PY@ul=\relax \let\PY@tc=\relax%
    \let\PY@bc=\relax \let\PY@ff=\relax}
\def\PY@tok#1{\csname PY@tok@#1\endcsname}
\def\PY@toks#1+{\ifx\relax#1\empty\else%
    \PY@tok{#1}\expandafter\PY@toks\fi}
\def\PY@do#1{\PY@bc{\PY@tc{\PY@ul{%
    \PY@it{\PY@bf{\PY@ff{#1}}}}}}}
\def\PY#1#2{\PY@reset\PY@toks#1+\relax+\PY@do{#2}}

\expandafter\def\csname PY@tok@nf\endcsname{\def\PY@tc##1{\textcolor[rgb]{0.00,0.00,1.00}{##1}}}
\expandafter\def\csname PY@tok@sr\endcsname{\def\PY@tc##1{\textcolor[rgb]{0.73,0.40,0.53}{##1}}}
\expandafter\def\csname PY@tok@cm\endcsname{\let\PY@it=\textit\def\PY@tc##1{\textcolor[rgb]{0.25,0.50,0.50}{##1}}}
\expandafter\def\csname PY@tok@mf\endcsname{\def\PY@tc##1{\textcolor[rgb]{0.40,0.40,0.40}{##1}}}
\expandafter\def\csname PY@tok@sd\endcsname{\let\PY@it=\textit\def\PY@tc##1{\textcolor[rgb]{0.73,0.13,0.13}{##1}}}
\expandafter\def\csname PY@tok@m\endcsname{\def\PY@tc##1{\textcolor[rgb]{0.40,0.40,0.40}{##1}}}
\expandafter\def\csname PY@tok@sb\endcsname{\def\PY@tc##1{\textcolor[rgb]{0.73,0.13,0.13}{##1}}}
\expandafter\def\csname PY@tok@ne\endcsname{\let\PY@bf=\textbf\def\PY@tc##1{\textcolor[rgb]{0.82,0.25,0.23}{##1}}}
\expandafter\def\csname PY@tok@gu\endcsname{\let\PY@bf=\textbf\def\PY@tc##1{\textcolor[rgb]{0.50,0.00,0.50}{##1}}}
\expandafter\def\csname PY@tok@c\endcsname{\let\PY@it=\textit\def\PY@tc##1{\textcolor[rgb]{0.25,0.50,0.50}{##1}}}
\expandafter\def\csname PY@tok@mi\endcsname{\def\PY@tc##1{\textcolor[rgb]{0.40,0.40,0.40}{##1}}}
\expandafter\def\csname PY@tok@gt\endcsname{\def\PY@tc##1{\textcolor[rgb]{0.00,0.27,0.87}{##1}}}
\expandafter\def\csname PY@tok@cpf\endcsname{\let\PY@it=\textit\def\PY@tc##1{\textcolor[rgb]{0.25,0.50,0.50}{##1}}}
\expandafter\def\csname PY@tok@vc\endcsname{\def\PY@tc##1{\textcolor[rgb]{0.10,0.09,0.49}{##1}}}
\expandafter\def\csname PY@tok@vi\endcsname{\def\PY@tc##1{\textcolor[rgb]{0.10,0.09,0.49}{##1}}}
\expandafter\def\csname PY@tok@vg\endcsname{\def\PY@tc##1{\textcolor[rgb]{0.10,0.09,0.49}{##1}}}
\expandafter\def\csname PY@tok@mh\endcsname{\def\PY@tc##1{\textcolor[rgb]{0.40,0.40,0.40}{##1}}}
\expandafter\def\csname PY@tok@ni\endcsname{\let\PY@bf=\textbf\def\PY@tc##1{\textcolor[rgb]{0.60,0.60,0.60}{##1}}}
\expandafter\def\csname PY@tok@s\endcsname{\def\PY@tc##1{\textcolor[rgb]{0.73,0.13,0.13}{##1}}}
\expandafter\def\csname PY@tok@nl\endcsname{\def\PY@tc##1{\textcolor[rgb]{0.63,0.63,0.00}{##1}}}
\expandafter\def\csname PY@tok@nc\endcsname{\let\PY@bf=\textbf\def\PY@tc##1{\textcolor[rgb]{0.00,0.00,1.00}{##1}}}
\expandafter\def\csname PY@tok@nn\endcsname{\let\PY@bf=\textbf\def\PY@tc##1{\textcolor[rgb]{0.00,0.00,1.00}{##1}}}
\expandafter\def\csname PY@tok@na\endcsname{\def\PY@tc##1{\textcolor[rgb]{0.49,0.56,0.16}{##1}}}
\expandafter\def\csname PY@tok@kn\endcsname{\let\PY@bf=\textbf\def\PY@tc##1{\textcolor[rgb]{0.00,0.50,0.00}{##1}}}
\expandafter\def\csname PY@tok@mo\endcsname{\def\PY@tc##1{\textcolor[rgb]{0.40,0.40,0.40}{##1}}}
\expandafter\def\csname PY@tok@sx\endcsname{\def\PY@tc##1{\textcolor[rgb]{0.00,0.50,0.00}{##1}}}
\expandafter\def\csname PY@tok@bp\endcsname{\def\PY@tc##1{\textcolor[rgb]{0.00,0.50,0.00}{##1}}}
\expandafter\def\csname PY@tok@ss\endcsname{\def\PY@tc##1{\textcolor[rgb]{0.10,0.09,0.49}{##1}}}
\expandafter\def\csname PY@tok@gs\endcsname{\let\PY@bf=\textbf}
\expandafter\def\csname PY@tok@si\endcsname{\let\PY@bf=\textbf\def\PY@tc##1{\textcolor[rgb]{0.73,0.40,0.53}{##1}}}
\expandafter\def\csname PY@tok@cp\endcsname{\def\PY@tc##1{\textcolor[rgb]{0.74,0.48,0.00}{##1}}}
\expandafter\def\csname PY@tok@no\endcsname{\def\PY@tc##1{\textcolor[rgb]{0.53,0.00,0.00}{##1}}}
\expandafter\def\csname PY@tok@cs\endcsname{\let\PY@it=\textit\def\PY@tc##1{\textcolor[rgb]{0.25,0.50,0.50}{##1}}}
\expandafter\def\csname PY@tok@s1\endcsname{\def\PY@tc##1{\textcolor[rgb]{0.73,0.13,0.13}{##1}}}
\expandafter\def\csname PY@tok@w\endcsname{\def\PY@tc##1{\textcolor[rgb]{0.73,0.73,0.73}{##1}}}
\expandafter\def\csname PY@tok@err\endcsname{\def\PY@bc##1{\setlength{\fboxsep}{0pt}\fcolorbox[rgb]{1.00,0.00,0.00}{1,1,1}{\strut ##1}}}
\expandafter\def\csname PY@tok@kd\endcsname{\let\PY@bf=\textbf\def\PY@tc##1{\textcolor[rgb]{0.00,0.50,0.00}{##1}}}
\expandafter\def\csname PY@tok@ch\endcsname{\let\PY@it=\textit\def\PY@tc##1{\textcolor[rgb]{0.25,0.50,0.50}{##1}}}
\expandafter\def\csname PY@tok@kr\endcsname{\let\PY@bf=\textbf\def\PY@tc##1{\textcolor[rgb]{0.00,0.50,0.00}{##1}}}
\expandafter\def\csname PY@tok@nd\endcsname{\def\PY@tc##1{\textcolor[rgb]{0.67,0.13,1.00}{##1}}}
\expandafter\def\csname PY@tok@sh\endcsname{\def\PY@tc##1{\textcolor[rgb]{0.73,0.13,0.13}{##1}}}
\expandafter\def\csname PY@tok@gr\endcsname{\def\PY@tc##1{\textcolor[rgb]{1.00,0.00,0.00}{##1}}}
\expandafter\def\csname PY@tok@ge\endcsname{\let\PY@it=\textit}
\expandafter\def\csname PY@tok@gi\endcsname{\def\PY@tc##1{\textcolor[rgb]{0.00,0.63,0.00}{##1}}}
\expandafter\def\csname PY@tok@gp\endcsname{\let\PY@bf=\textbf\def\PY@tc##1{\textcolor[rgb]{0.00,0.00,0.50}{##1}}}
\expandafter\def\csname PY@tok@k\endcsname{\let\PY@bf=\textbf\def\PY@tc##1{\textcolor[rgb]{0.00,0.50,0.00}{##1}}}
\expandafter\def\csname PY@tok@kp\endcsname{\def\PY@tc##1{\textcolor[rgb]{0.00,0.50,0.00}{##1}}}
\expandafter\def\csname PY@tok@ow\endcsname{\let\PY@bf=\textbf\def\PY@tc##1{\textcolor[rgb]{0.67,0.13,1.00}{##1}}}
\expandafter\def\csname PY@tok@kc\endcsname{\let\PY@bf=\textbf\def\PY@tc##1{\textcolor[rgb]{0.00,0.50,0.00}{##1}}}
\expandafter\def\csname PY@tok@s2\endcsname{\def\PY@tc##1{\textcolor[rgb]{0.73,0.13,0.13}{##1}}}
\expandafter\def\csname PY@tok@sc\endcsname{\def\PY@tc##1{\textcolor[rgb]{0.73,0.13,0.13}{##1}}}
\expandafter\def\csname PY@tok@gh\endcsname{\let\PY@bf=\textbf\def\PY@tc##1{\textcolor[rgb]{0.00,0.00,0.50}{##1}}}
\expandafter\def\csname PY@tok@nb\endcsname{\def\PY@tc##1{\textcolor[rgb]{0.00,0.50,0.00}{##1}}}
\expandafter\def\csname PY@tok@nv\endcsname{\def\PY@tc##1{\textcolor[rgb]{0.10,0.09,0.49}{##1}}}
\expandafter\def\csname PY@tok@kt\endcsname{\def\PY@tc##1{\textcolor[rgb]{0.69,0.00,0.25}{##1}}}
\expandafter\def\csname PY@tok@nt\endcsname{\let\PY@bf=\textbf\def\PY@tc##1{\textcolor[rgb]{0.00,0.50,0.00}{##1}}}
\expandafter\def\csname PY@tok@mb\endcsname{\def\PY@tc##1{\textcolor[rgb]{0.40,0.40,0.40}{##1}}}
\expandafter\def\csname PY@tok@go\endcsname{\def\PY@tc##1{\textcolor[rgb]{0.53,0.53,0.53}{##1}}}
\expandafter\def\csname PY@tok@se\endcsname{\let\PY@bf=\textbf\def\PY@tc##1{\textcolor[rgb]{0.73,0.40,0.13}{##1}}}
\expandafter\def\csname PY@tok@gd\endcsname{\def\PY@tc##1{\textcolor[rgb]{0.63,0.00,0.00}{##1}}}
\expandafter\def\csname PY@tok@il\endcsname{\def\PY@tc##1{\textcolor[rgb]{0.40,0.40,0.40}{##1}}}
\expandafter\def\csname PY@tok@o\endcsname{\def\PY@tc##1{\textcolor[rgb]{0.40,0.40,0.40}{##1}}}
\expandafter\def\csname PY@tok@c1\endcsname{\let\PY@it=\textit\def\PY@tc##1{\textcolor[rgb]{0.25,0.50,0.50}{##1}}}

\def\PYZbs{\char`\\}
\def\PYZus{\char`\_}
\def\PYZob{\char`\{}
\def\PYZcb{\char`\}}
\def\PYZca{\char`\^}
\def\PYZam{\char`\&}
\def\PYZlt{\char`\<}
\def\PYZgt{\char`\>}
\def\PYZsh{\char`\#}
\def\PYZpc{\char`\%}
\def\PYZdl{\char`\$}
\def\PYZhy{\char`\-}
\def\PYZsq{\char`\'}
\def\PYZdq{\char`\"}
\def\PYZti{\char`\~}
% for compatibility with earlier versions
\def\PYZat{@}
\def\PYZlb{[}
\def\PYZrb{]}
\makeatother


    % Exact colors from NB
    \definecolor{incolor}{rgb}{0.0, 0.0, 0.5}
    \definecolor{outcolor}{rgb}{0.545, 0.0, 0.0}


    \newcommand{\genericdel}[3]{%
      \left#1#3\right#2
    }
    \newcommand{\del}[1]{\genericdel(){#1}}
    \newcommand{\sbr}[1]{\genericdel[]{#1}}
    \newcommand{\cbr}[1]{\genericdel\{\}{#1}}
    \DeclareMathOperator*{\argmin}{arg\,min}
    \DeclareMathOperator*{\argmax}{arg\,max}
	\let\Pr\relax
    \DeclareMathOperator{\Pr}{\mathbb{P}}
    \DeclareMathOperator{\E}{\mathbb{E}}
    \DeclareMathOperator{\V}{\mathbb{V}}
    \DeclareMathOperator{\cov}{{cov}}
    \DeclareMathOperator{\var}{{var}}
    \DeclareMathOperator{\Ind}{\mathbf{I}}
	\DeclareMathOperator*{\sgn}{{sgn}}

    \DeclareMathOperator{\normal}{\mathcal{N}}
    \DeclareMathOperator{\unif}{Uniform}
    \DeclareMathOperator{\invchi}{\mathrm{Inv-\chi}^2}

    \newcommand{\abs}[1]{\genericdel||{#1}}
    \newcommand{\effect}{\mathrm{eff}}
    \newcommand{\xtilde}{\widetilde{X}}
    \newcommand{\boxleft}{\squareleftblack}
    \newcommand{\boxright}{\squarerightblack}
    \newcommand{\discont}{\boxbar}
    \newcommand{\jleft}{\unicode{x21E5}}
    \newcommand{\jright}{\unicode{x21E4}}

    \DeclareMathOperator{\ones}{\mathbf{1}}
    \DeclareUnicodeCharacter{9707}{$\boxbar$}
    
    \DeclareMathOperator{\GP}{\mathcal{GP}}
    \DeclareMathOperator{\scrl}{\mathscr{l}}
    \newcommand{\saleprice}{\mathtt{SalePrice}}
    \newcommand{\sqft}{\mathtt{SQFT}}
    \newcommand{\xvec}{\mathbf{x}}
    \newcommand{\tax}{\mathtt{TaxClass}}
    \newcommand{\building}{\mathtt{BuildingClass}}

    \newcommand{\gp}{\mathcal{GP}}
    \newcommand{\trans}{^{\intercal}}
    \newcommand{\scrS}{\mathscr{S}}
    \newcommand{\sigmaf}{\sigma_{\mathrm{GP}}}
    \newcommand{\sigman}{\sigma_{\epsilon}}
    \newcommand{\sigmatau}{\sigma_{\tau}}
    \newcommand{\sigmabeta}{\sigma_{\beta}}
    \newcommand{\sigmamu}{\sigma_{\mu}}
    \newcommand{\sigmagamma}{\sigma_{\gamma}}
    \newcommand{\svec}{\mathbf{s}}
    \newcommand{\indep}{\perp}
    \newcommand{\iid}{iid}
    \newcommand{\vectreat}{\Ind_{T}}

    \newcommand{\boundary}{\partial}
    \newcommand{\sentinels}{\bm{\boundary}}
    \newcommand{\eye}{\mathbf{I}}
    \newcommand{\K}{\mathbf{K}}

	\providecommand{\tightlist}{%
  	  \setlength{\itemsep}{0pt}\setlength{\parskip}{0pt}}


    
    % Prevent overflowing lines due to hard-to-break entities
    \sloppy 
    % Setup hyperref package
    \hypersetup{
      breaklinks=true,  % so long urls are correctly broken across lines
      colorlinks=true,
      urlcolor=urlcolor,
      linkcolor=linkcolor,
      citecolor=citecolor,
      }
    % Slightly bigger margins than the latex defaults
    
    \geometry{verbose,tmargin=1in,bmargin=1in,lmargin=1in,rmargin=1in}
    
    

    \begin{document}
    
    
    
    \maketitle
    
    
	\tableofcontents


    




    	\section{Introduction}\label{introduction}

\subsection{Motivation}\label{motivation}

\subsection{Prior attempts}\label{prior-attempts}
    


    	\section{Model Specification}\label{model-specification}

\subsection{Notation}\label{notation}

\begin{itemize}
\tightlist
\item
  2-dimensional coordinate space \(\scrS\)
\item
  treatment units are in region \(\scrS_T \subset \scrS\) and control
  units are in non-overlapping \(\scrS_C\) outside of the treatment
  region, so that \(\scrS_C = \scrS_T^c\) and
  \(\scrS_T \cup \scrS_C = \scrS\)
\item
  Observed outcomes for units in treatment region \(s \in \scrS_T\) are
  labeled \(Y_T(\svec)\), and units in control region \(Y_C(\svec)\).
\item
  Potential outcomes framework: Each unit has a potential outcome under
  treatment \(Y_T(\svec)\) and a potential outcome under control
  \(Y_C(\svec)\). If \(s \in \scrS_T\), then \(Y_T(\svec)\) is observed,
  otherwise \(Y_C(\svec)\) is observed.
\end{itemize}

\subsection{1GP solution}\label{gp-solution}

Most straightforwardly, we model the observed outcomes \(Y\) at
locations \(S\) as the sum of an intercept \(\mu\), linear trend
\(S\beta\), a spatial Gaussian process \(f(S)\), a constant treatment
effect \(\tau\) in the treatment region, and iid normal noise
\(\epsilon\).

\begin{align}
Y_i(\svec) &= \mu+\svec\trans\beta + f(\svec) + \tau \Ind\cbr{\svec \in \scrS_T} + \epsilon_i \\
f(S) &\sim \gp\del{0, k(\svec, \svec')} \\
k(\svec,\svec') &= \sigmaf^2 \exp\del{ - \frac{\del{\svec-\svec'}\trans\del{\svec-\svec'}}{2 \ell^2}} \\
\epsilon_i &\overset{\iid}{\sim} \normal\del{0,\sigma_\epsilon^2}
\end{align}

\(f(S)\) is a smooth surface covering all of \(\scrS\), specificed as a
Gaussian Process with squared exponential covariance kernel \(k\) with
lengthscale \(\ell\) and variance \(\sigmaf^2\). The squared exponential
kernel is frequently used in spatial settings. The constant treatment
effect implies the assumption that \(Y_T(\svec) = \tau + Y_C(\svec)\)
for all units at all locations.

\subsection{2GP solution}\label{gp-solution-1}

The constant treatment effect is a strong assumption that will be hard
to justify in many applications. To allow the treatment effect to vary
spatially, an alternative is to specify two independent Gaussian
processes for the treatment response and the control response.

\begin{align}
Y_{T,i}(\svec) &= \mu_T + \svec\trans\beta_T + f_T(\svec) + \epsilon_i \\
Y_{C,i}(\svec) &= \mu_C + \svec\trans\beta_C + f_C(\svec) + \epsilon_i \\
f_T(S), f_C(S) &\overset{\indep}{\sim} \gp\del{0, k(\svec, \svec')} \\
k(\svec,\svec') &= \sigmaf^2 \exp\del{ - \frac{\del{\svec-\svec'}\trans\del{\svec-\svec'}}{2 \ell^2}} \\
\end{align}

Here, the treatment effect \(\tau\) is no longer included explicitly in
the model. Instead, the treatment effect at a location \(\svec\) is
derived as the difference between the two (noise-free) surfaces.

\[
\tau(\svec) = \sbr{\mu_T + \svec\trans\beta_T + f_T(\svec)} - \sbr{\mu_C + \svec\trans\beta_C + f_C(\svec)}
\]

In this specification, the kernel parameters \(\ell\) and \(\sigmaf\)
are the same in the treatment and control regions, so we assume that the
spatial smoothness of the responses isn't affected by the treatment.
This assumption will be reasonable in most applications, but can be
easily relaxed. Inference on the hyperparameters proceeds as in the 1GP
case, using the sum of the likelihood in the treatment and control
regions.

\subsection{Discussion}\label{discussion}

\begin{itemize}
\tightlist
\item
  different assumptions
\item
  will stick to 2GP from now on
\end{itemize}
    


    	\section{Inference}\label{inference}

By specifying the spatial variation as Gaussian processes, we can
leverage the properties of multivariate normals to obtain analytical
forms for the estimate of the treatment effect.
    


    	\subsection{1GP}\label{gp}

We proceed by placing normal priors on \(\mu\), \(\beta\) and \(\tau\).
The model specification can then be used to obtain covariances between
the observations and these parameters. In fact,
\(\del{Y,f(S),\tau,\mu,\beta} \mid \ell,\sigmaf\) is multi-variate
normal with variance-covariance given by

\begin{align}
    \tau  &\sim \normal\del{0,\sigmatau^2} \\
    \mu   &\sim \normal\del{0,\sigmamu^2} \\
    \beta &\sim \normal\del{0,\sigmabeta^2} \\
    \cov(Y_i(\svec),\tau) &= \sigmatau^2 \Ind\cbr{\svec \in \scrS_T} \\
    \cov(Y_i(\svec),\mu)  &= \sigmamu^2 \\
    \cov(Y_i(\svec),\beta)&= \sigmabeta^2 \svec\trans \svec \\
    \cov(Y_i(\svec),Y_i(\svec'))&= \sigmamu^2 + \sigmatau^2 \Ind\cbr{\svec \in \scrS_T}\Ind\cbr{\svec' \in \scrS_T} + \sigmabeta^2 \svec\trans \svec' + k(\svec,\svec') + \delta_{ij}\sigman^2\\
    \cov(Y(\svec),f(\svec')) = \cov(f(\svec),f(\svec')) &= k(\svec,\svec')
\end{align}

Multi-variate theory then allows us to condition any of these objects on
the others. We are particularly interested in the posterior distribution
\(\tau \mid Y, \ell, \sigmaf\) which is given by \[
    \tau \mid Y, \ell, \sigmaf \sim \normal\del{\cov\del{Y,\tau}\trans \cov\del{Y}^{-1} Y, \sigma_\tau^2 - \cov\del{Y,\tau}\trans \cov\del{Y}^{-1} \cov\del{Y,\tau}}
\]

To proceed computationally, we define the treatment indicator vector
\(\vectreat\) with \(i\)th entry equal to 0 when \(\svec_i\) is in the
control region, and 1 in the treatment region, and the \(n \times n\)
kernel covariance matrix \(\K\) having entries
\(K_{ij}=k(\svec_i, \svec_j)\). The posterior mean and variance are then
easily computed.

\begin{align}
    \E \del{\tau \mid Y, \ell, \sigmaf} &= \sigmatau^2 \vectreat\trans \cbr{\sigmamu^2 + \sigmatau^2 \vectreat \vectreat\trans + \sigmabeta^2 S S\trans + \K + \sigman^2 \eye }^{-1} Y \\
    \var \del{\tau \mid Y, \ell, \sigmaf} &= \sigma_\tau^2 - \sigma_\tau^2 \vectreat\trans \cbr{\sigmamu^2 + \sigmatau^2 \vectreat \vectreat\trans + \sigmabeta^2 S S\trans + \K + \sigman^2 \eye }^{-1} \vectreat
\end{align}

What remains is the inference on the hyperparameters
\(\sigman, \sigmaf\) and \(\ell\). The two approaches typically taken in
modern spatial statistics are either to maximize the marginal likelihood
of \(Y\) as a function of those three parameters, or to assign them a
prior and take a Bayesian approach, requiring that the posterior of
\(\tau\) be integrated over those parameters. The compromise is clear:
the Bayesian approach incorporates the uncertainty in the
hyperparameters, thus giving more reliable inference on \(\tau\), but
maximizing the marginal likelihood has a much lower computation cost.
Therefore, we recommend taking the Bayesian approach whenever
computationally possible, and maximizing the marginal likelihood when
the data is larger.
    


    	\subsection{2GP}\label{gp}

In the 2GP setting, we begin by modeling the treatment and control units
with two independent Gaussian processes with shared hyperparameters.
Because the treatment and control regions do not overlap, inference on
the treatment effect is only measurable near the boundary. In the
classical one-dimensional regression discontinuity design, the estimand
is therefore defined at the boundary \(x=b\):
\[\tau = \lim_{x \downarrow b} \E\sbr{y \mid X=s} - \lim_{x \uparrow b} \E\sbr{y \mid X=x} = \E\sbr{Y_T \mid X=b} - \E\sbr{Y_C \mid X=b}\]
Analogously, we focus on the treatment effect at the boundary
\(\boundary\) between the treatment and control regions. \(\boundary\)
is therefore a one-dimensional subset of \(\scrS\). We will proceed by
extrapolating both Gaussian processes to the boundary, and then
subtracting the predictions to obtain the estimated treatment effect.
Computationally, we need to represent this boundary as a set of \(k\)
``sentinel'' units distributed along the boundary
\(\sentinels=\cbr{\boundary_1,\ldots,\boundary_k},~\partial_i \in \partial\).
    


    	\section{Handling covariates}\label{handling-covariates}

The Gaussian Process specification makes it easy to incorporate a linear
model on non-spatial covariates, both mathematically and
computationally. The model is modified by the addition of the linear
regression term \(D \gamma\) on the \(n \times p\) matrix of covariates
\(D\). In the spirit of ridge regression, we recommend placing a normal
prior \(\normal(0,\sigmagamma^2)\) on the regression coefficients. This
preserves the multivariate normality of the problem, with the simple
addition of a term \(\sigmagamma^2 D\trans D\) to the covariance of
\(Y\).

With the 1GP model, covariates can therefore be handled at very little
additional cost, except that the additional hyperparameter
\(\sigmagamma^2\) needs to be fitted.
    


    	\section{2GP: Testing for non-zero
effect}\label{gp-testing-for-non-zero-effect}

\section{Average treatment effect}\label{average-treatment-effect}

\subsection{Linear average}\label{linear-average}

\subsection{Inverse variance}\label{inverse-variance}

\section{Spatial advantage}\label{spatial-advantage}

\begin{itemize}
\tightlist
\item
  spreading units along a boundary doesn't necessarily reduce power
\item
  multiple experiments interpretation
\end{itemize}

\section{Example: NYC school
districts}\label{example-nyc-school-districts}

\section{Conclusion}\label{conclusion}
    



    % Add a bibliography block to the postdoc
    
    
    
    \end{document}
