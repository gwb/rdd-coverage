The calibrated inverse-variance test of \autoref{sec:hypothesis_testing} is the special case of this procedure with weights \(\weightb(\sentinels) = \Sigma^{-1}_{\sentinels \mid Y} \ones_{\numsent}\).

\section{Covariances for Gaussian Process Model}
\label{sec:covariances}
All covariances below are conditional on the hyperparameters \(\hyperparam = \del{\ell,\sigmaf, \sigman, \sigmamu}\), omitted for concision.
We further define some shorthand notation, found in \autoref{table:notation}.
\begin{equation}
    \begin{split}
        m_\treat, m_\ctrol   &\sim \normal\del{0,\sigmamu^2} \\
        \cov(Y_{i\treat},m_\treat)  &= \cov(Y_{i\ctrol},m_\ctrol) = \sigmamu^2 \\
        \cov(Y_{i\treat},m_\ctrol)  &= \cov(Y_{i\ctrol},m_\treat)  = 0 \\
        \cov\del{Y_{i\treat}, f_{\treat}(\svec')} &= \cov\del{Y_{i\ctrol}, f_{\ctrol}(\svec')} = k(\svec_i,\svec') \\
        \cov\del{Y_{i\treat}, f_{\ctrol}(\svec')} &= \cov\del{Y_{i\ctrol}, f_{\treat}(\svec')} = 0 \\
        \cov(Y_{i\treat},Y_{j\treat}) &= \cov(Y_{i\ctrol},Y_{j\ctrol}) = \sigmamu^2 + k(\svec_i,\svec_j) + \delta_{ij}\sigman^2 \\
        \cov(Y_{i\treat},Y_{j\ctrol}) &= 0
    \end{split}
    \label{eq:covariances}
\end{equation}

\begin{table}[bp]
    \centering
    \bgroup
    \def\arraystretch{1.2}%  1 is the default, change whatever you need
    \begin{tabular}{lll}
        \hline
        Symbol & Size                       & \(ij^{\mathrm{th}}\) entry                                                      \\ \hline
        \(\KBB\) & \(\numsent \times \numsent\) & \(\sigmamu^2 + k\del{\sentinel_i,\sentinel_j}\)                                 \\ 
        \(\KBT\) & \(\numsent \times n_\treat\) & \(\sigmamu^2 + k\del{\sentinel_i,\svec_{j\treat}}\)                             \\ 
        \(\KBC\) & \(\numsent \times n_\ctrol\) & \(\sigmamu^2 + k\del{\sentinel_i,\svec_{j\ctrol}}\)                             \\
        \(\KTT\) & \(n_\treat \times n_\treat\) & \(\sigmamu^2 + k\del{\svec_{i\ctrol},\svec_{j\ctrol}}\)                         \\
        \(\KCC\) & \(n_\ctrol \times n_\ctrol\) & \(\sigmamu^2 + k\del{\svec_{i\treat},\svec_{j\treat}}\)                         \\ 
        \(\STT\) & \(n_\treat \times n_\treat\) & \(\sigmamu^2 + k\del{\svec_{i\treat},\svec_{j\treat}} + \delta_{ij} \sigman^2\) \\ 
        \(\SCC\) & \(n_\ctrol \times n_\ctrol\) & \(\sigmamu^2 + k\del{\svec_{i\ctrol},\svec_{j\ctrol}} + \delta_{ij} \sigman^2\) \\
        \hline
    \end{tabular}
    \egroup
    \caption{
        Shorthand notation for covariance matrices. The spatial coordinates of the \(i^\mathrm{th}\) treatment unit are denoted by \(\svec_{i\treat}\),
and those of the \(j^\mathrm{th}\) control unit by \(\svec_{j\ctrol}\), while \(\sentinel_i\) denotes the \(i^\mathrm{th}\) sentinel location along the border.
        \label{table:notation}
    }
\end{table}


\section{Calibration of Inverse-variance Test}
\label{sec:calibration}

We seek to obtain a valid hypothesis test against the null hypothesis of zero treatment effect everywhere along the border by using the inverse-variance weighted LATE estimate \autoref{eq:invvar} as a test statistic.

Under the parametric null hypothesis \(\modnull\), \(\yt\) and \(\yc\) are drawn from a single Gaussian process, with no discontinuity at the border.
Their joint covariance is
\begin{equation}
\cov \del{\begin{pmatrix}\yt \\ \yc\end{pmatrix} \mid \modnull } 
    = \begin{bmatrix}
        \STT & \KTC \\
        \KTC \trans & \SCC
    \end{bmatrix}\,,
\end{equation}
where \(\KTC\) is the \(n_\treat \times n_\ctrol\) matrix with \(ij^{\mathrm{th}}\) entry equal to \(k\del{\svec_{i\treat},\svec_{j\ctrol}}\).
The predicted mean outcomes \autoref{eq:postvar2gp_t_or_c} at the sentinels \(\muvec_{\sentinels \mid T}\) and \(\muvec_{\sentinels \mid T}\) are obtained by left-multiplying \(\yt\) and \(\yc\) by matrices \(\WT\) and \(\WC\) (respectively) that are deterministic functions of the unit locations and the hyperparameters:
\begin{equation}
    \WT = \KBT \STT^{-1} \quad\text{and}\quad
    \WC = \KBC \SCC^{-1}\,.
\end{equation}

Under \(\modnull\), the joint distribution of \(\muvec_{\sentinels \mid T}\) and \(\muvec_{\sentinels \mid T}\) is consequently also multivariate normal with mean zero and covariance given by:
\begin{equation}
\cov \del{\begin{pmatrix}\WT \yt \\ \WC \yc \end{pmatrix} \mid \modnull } = \begin{bmatrix}
    \WT \STT       \WT\trans & \WT \KTC \WC\trans \\
    \WC \KTC\trans \WT\trans & \WC \SCC \WC\trans
\end{bmatrix}\,.
\end{equation}
Continuing in this fashion, the cliff height \autoref{eq:postvar2gp} estimate 
\(\muvec_{\sentinels \mid Y} = \WT \yt - \WC \yc\)
is yet another zero-mean multivariate normal with covariance given by:
\begin{equation}
        \cov \del{\muvec_{\sentinels \mid Y} \mid \modnull} 
        = \WT \STT \WT\trans + \WC \SCC \WC\trans - \WT \KTC \WC\trans -  \WC\KTC\trans\WT\trans \,.
\end{equation}

Weighted LATE estimators of the form defined in \autoref{eq:weighted_estimator} are linear transformations of \(\muvec_{\sentinels \mid Y}\) and so under \(\modnull\), they are normally distributed with mean zero.
For a given weight function \(\weightb\), its variance is given by
\begin{equation}
        \var\del{\mu_{\tauw \mid Y} \mid \modnull} = \cov\del{ \frac{\weightb(\sentinels)\trans \muvec_{\sentinels \mid Y}}{ \weightb(\sentinels)\trans\ones_{\numsent}}}
        = \frac{\weightb(\sentinels) \trans \cov \del{\muvec_{\sentinels \mid Y}} \weightb(\sentinels)}{\del{ \weightb(\sentinels)\trans\ones_{\numsent}}^2}
        \,.
\end{equation}

The \(p\)-value follows from treating the LATE estimate as a test statistic.
Under the null hypothesis, the probability of \(\mu_{\tauw \mid Y}\) exceeding in magnitude its observed value \(\mu_{\tauw \mid Y^{obs}}\) is:
\begin{equation}
    \prob\del{ \abs{\mu_{\tauw \mid Y}} \ge \abs{\mu_{\tauw \mid Y^{obs}}} \mid \modnull} = 2 \Phi\del{ -\abs{\mu_{\tauw \mid Y^{obs}}} \big/ {\sqrt{\var({\mu_{\tauw \mid Y} \mid \modnull})}} }\,.
\end{equation}
The calibrated inverse-variance test of \autoref{sec:hypothesis_testing} is the special case of this procedure with weights \(\weightb(\sentinels) = \Sigma^{-1}_{\sentinels \mid Y} \ones_{\numsent}\).
