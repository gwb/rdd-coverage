What is the premium on house price for a particular school district?
To estimate this in New York City we use a novel implementation of a Geographic Regression Discontinuity Design (GeoRDD) built from Gaussian processes regression (kriging) to model spatial structure.
With a GeoRDD, we specifically examine price differences along borders between ``treatment'' and ``control'' school districts.
GeoRDDs extend RDDs to multivariate settings; location is the forcing variable and the border between school districts constitutes the discontinuity threshold.
We first obtain a Bayesian posterior distribution of the price difference function, our nominal treatment effect, along the border.
We then address nuances of having a functional estimand defined on a border with potentially intricate topology, particularly when defining and estimating causal estimands of the local average treatment effect (LATE).
We test for nonzero LATE with a calibrated hypothesis test with good frequentist properties, which we further validate using a placebo test.
Using our methodology, we identify substantial differences in price across several borders.
In one case, a border separating Brooklyn and Queens, we estimate a statistically significant 20\% higher price for a house on the more desirable side.
We also find that geographic features can undermine some of these comparisons.
