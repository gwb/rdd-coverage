Most research on regression discontinuity designs (RDDs) has focused on univariate cases, where only those units with a ``forcing'' variable on one side of a threshold value receive a treatment.
Geographical regression discontinuity designs (GeoRDDs) extend the RDD to multivariate settings with spatial forcing variables.
We propose a framework for analysing GeoRDDs, which we implement using Gaussian process regression. 
This yields a Bayesian posterior distribution of the treatment effect at every point along the border.
We address nuances of having a functional estimand defined on a border with potentially intricate topology, particularly when defining and estimating causal estimands of the local average treatment effect (LATE).
The Bayesian estimate of the LATE can also be used as a test statistic
in a hypothesis test with good frequentist properties, 
which we validate using placebo tests.
We demonstrate our methodology with a dataset of property sales in New York City,
to assess whether there is a discontinuity in housing prices at the border between two school districts.
We find a statistically significant difference in price across the border between the districts with \(p\)=0.002, and estimate a 20\%  higher price on average for a house on the more desirable side.
